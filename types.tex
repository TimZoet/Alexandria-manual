A new type can be created with the library's \code{createType} method, which returns a reference to the new \gls{type_class} object. The reference is valid as long as the library has not been deallocated. You can also retrieve it by name with the \code{getType} method. See Listing~\ref{lst:types:type_create}.

\lstinputlisting[caption={Creating and getting a type.}, label={lst:types:type_create}]{snippets/type_create.cpp}

There are some restrictions on type names. Obviously, you cannot create two types with the same name. Additionally, there are some forbidden prefixes to prevent collisions with the database's internal and generated tables. These prefixes are: %TODO: List forbidden prefixes.

To specify the data the type will hold, you can add properties to it using the \code{create*Property} methods. See Listing~\ref{lst:types:type_properties}. For a description of all property types, see Section~\ref{section:types:property_types}.

\lstinputlisting[caption={Adding some properties.}, label={lst:types:type_properties}]{snippets/type_properties.cpp}

Now, some details on what happens internally in terms of what kind of tables are created. For each type that you create, one or more tables are generated. At the very least, the main \gls{instance_table} is created. This table will contain a column for each property that can be stored as a single value. When creating a new instance if this type, a new row is added with all the values. Additional tables are generated for e.g. arrays of values or references to other objects.

To insert a type in the database and generate all tables, you can either call the \code{commitType} method to add a specific type or the \code{commitTypes} to add all not yet committed types. See Listing~\ref{lst:types:type_commit}. After committing a type, it cannot be modified. Note that if the library goes out of scope before a type is committed, it is lost. Also, when committing a single type other types on which it depends will be committed as well.

\lstinputlisting[caption={Committing types.}, label={lst:types:type_commit}]{snippets/type_commit.cpp}

\subsection{Property Types}
\label{section:types:property_types}

%TODO: Blob values and endianness/portability.

All types are automatically given a 64-bit signed integer property. This property is used as the instance identifier. Internally, it is used as the primary key of the database table. Additionally, you can define any number of differently typed properties, ranging from integers to raw binary data.

Just as with the type names, there are restrictions on property names. These are: %TODO: Forbidden property names.

The remaining subsections describe all supported property types. Section~\ref{section:classes} explains how to implement C++ classes to match your type definitions.

\subsubsection{Primitives}
\label{section:types:property_types:primitives}

Primitive properties can be used to store integer and floating point values. They can be added to a type using the \code{createPrimitiveProperty}. See Listing~\ref{lst:types:property_primitive}.

\lstinputlisting[caption={Creating primitive properties.}, label={lst:types:property_primitive}]{snippets/property_primitive.cpp}

The \code{setDefaultValue} method can be used to change the default value of each property. See Listing~\ref{lst:types:property_primitive_default}.

\lstinputlisting[caption={Setting default values.}, label={lst:types:property_primitive_default}]{snippets/property_primitive_default.cpp}

Primitive values are stored in the instance table as a single column value. It is also possible to create arrays of primitive values using the \code{createPrimitiveArrayProperty}. These are stored in a separate table. This table has a column with a reference to the instance table and a column with a primitive value. For each element of the primitive array, a row is added. See Listing~\ref{lst:types:property_primitive_array}.

\lstinputlisting[caption={Creating primitive array properties.}, label={lst:types:property_primitive_array}]{snippets/property_primitive_array.cpp}

While primitive arrays are useful, storing values in a separate table with a row for each value can be rather inefficient for large arrays (where large is anything above a few dozen values). To solve this problem, there are primitive blobs. These store arrays of primitive values as a contiguous binary object in the instance table. See Listing~\ref{lst:types:property_primitive_blob}.

\lstinputlisting[caption={Creating primitive blob properties.}, label={lst:types:property_primitive_blob}]{snippets/property_primitive_blob.cpp}

Section~\ref{section:member_types:primitive_array} and Section~\ref{section:member_types:primitive_blob} describe the wrapper classes that manage primitive arrays and blobs.

\subsubsection{Strings}
\label{section:types:property_types:strings}

String properties can be used to store single strings and arrays of strings. They can be added to a type using the \code{createStringProperty} and \code{createStringArrayProperty} methods. See Listing~\ref{lst:types:property_string}.

\lstinputlisting[caption={Creating string properties.}, label={lst:types:property_string}]{snippets/property_string.cpp}

Single string values are stored in the instance table as a single column value. For arrays of strings, a separate table is generated. This table has a column with a reference to the instance table and a column with a string value. For each element of the string array, a row is added.

Section~\ref{section:member_types:string} and Section~\ref{section:member_types:string_array} describe the wrapper classes that manage string values and arrays.

\subsubsection{Blobs}
\label{section:types:property_types:blobs}

Blob properties can be used to store arbitrary binary data. No type information is maintained by \gls{alexandria}. This is entirely the responsibility of the user. Note that the primitive blob property described in Section~\ref{section:types:property_types:primitives} also stores binary data, but that property does store type information.

There are two types of blob properties: single and array. They can be added to a type using the \code{createBlobProperty} and \code{createBlobArrayProperty} methods. See Listing~\ref{lst:types:property_blob}.

\lstinputlisting[caption={Creating blob properties.}, label={lst:types:property_blob}]{snippets/property_blob.cpp}

Single blob values are stored in the instance table as a single column value. For arrays of blobs, a separate table is generated. This table has a column with a reference to the instance table and a column with a blob value. For each element of the blob array, a row is added.

Section~\ref{section:member_types:blob} and Section~\ref{section:member_types:blob_array} describe the wrapper classes that manage blob values and arrays.

\subsubsection{References}
\label{section:types:property_types:references}

%TODO: References to same type.

Reference properties can be used to store references to instances of some other type. They can be added to a type using the \code{createReferenceProperty} and \code{createReferenceArrayProperty}. Both these methods take a reference to an already created type. See Listing~\ref{lst:types:property_reference}.

\lstinputlisting[caption={Creating reference properties.}, label={lst:types:property_reference}]{snippets/property_reference.cpp}

Single reference values are stored in the instance table as a single column value. More specifically, the identifier of the referenced instance is stored. For arrays of references, a separate table is generated. This table has a column with a reference to the instance table and a column with a reference to the referenced type. For each referenced instance in the reference array, a row is added.

\subsubsection{Nested Types}
\label{section:types:property_types:types}

To make defining complex types a bit easier, it is possible to create type properties. This is essentially the equivalent of having one class as a member variable of another in C++. Types can be added to any number of other types. Adding a type property that would result in a circular dependency is not allowed. You can repeatedly add the same type (with a different name, of course). See Listing~\ref{lst:types:property_nested_type}.

\lstinputlisting[caption={Creating nested properties.}, label={lst:types:property_nested_type}]{snippets/property_nested_type.cpp}

Adding a type property effectively calls the property creation methods for the type that you're adding. In the above example, the \code{transform} type would end up with 9 float columns in the instance table.